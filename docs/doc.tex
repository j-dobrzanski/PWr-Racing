\documentclass{article}
\usepackage[textheight=10in]{geometry}
\title{Doc}
\author{Geometry (figury)}
\date{}
\usepackage{float}
\usepackage{polski}
\usepackage{amsmath}
\usepackage{graphicx}
\usepackage{algorithm}
\usepackage[noend]{algpseudocode}

\begin{document}
\maketitle
\section{General Info}
There are some techniques that can be used
to achieve pseudo-OOP in C. Here standard classes are split in two parts:
\begin{itemize}
	\item fields of a class are kept in a structure of the same name as a class would be named,
	\item methods are functions grouped by their names to match class name with certain prefixes and always take structure pointer as a parameter
\end{itemize}  
Moreover abstraction in OOP can be simulated to some extent by adding additional field describing which exact object must be dealt with.
In this way, as stated in task, structures are:
\begin{itemize}
	\item struct geometry\textunderscore point \{ double x; double y;\};
	\item struct geometry\textunderscore segment \{ geometry\textunderscore point* start; geometry\textunderscore point* end;\};
	\item struct geometry\textunderscore triangle \{ geometry\textunderscore point* first; geometry\textunderscore point* second; geometry\textunderscore point* third; bool is\textunderscore right;\};
\end{itemize}
Whole task is implemented as a C library with only API visible, so forward-declaration of structures are used.
To enable user to interact with structures some simple constructors/destructors/getters are also added.

\section{Functions implementations background}
\subsection{Moving by vector}
Moving points on a coordinate plane by a vector with initial point in (0,0) is a simple task - one should just add coordinates of terminal point of this vector to coordinates of the point. If one wants to move whole shape described by some finite amount of points then all points can be moved one by one.
\subsection{Rotating through an angle}
If given is an angle $ \phi $ measured in radians and calculated counterclockwise then this calculation can be pretty simple too, assuming rotation around (0,0) just use trigonometric functions: 
$$ (x,y) \rightarrow  (x\cos\phi-y\sin\phi, x\sin\phi+y\cos\phi) $$
If rotation around some specified point $ (x_0,y_0) $ is need, then moving plane by vector forth and back can be applied:
$$ (x,y) \rightarrow ((x-x_0)\cos\phi-(y-y_0)\sin\phi + x_0, (x-x_0)\sin\phi+(y-y_0)\cos\phi + y_0) $$
Once again - if one wants to rotate whole shape described by some finite amount of points then all points can be rotated one by one.
\subsection{Calculating distances}
Distance between two points $ A = (x_1,y_1) $ and $ B = (x_2,y_2) $ in euclidean space can be calculated with equation 
$$ |AB| = \sqrt{x_1-x_2)^2+(y_1-y_2)^2} $$
Of course length of a line segment is basically a distance between two endpoints and perimeter of a triangle is a sum of all sides length
\subsection{Point in the segment}
TBD
\subsection{Parallel/perpendicular segments}
With two segements given, $AB = \{(x_A,y_A),(x_B,y_B)\}, CD = \{(x_C,y_C),(x_D,y_D)\} $, one can determine if they are parallel/perpendicular by comparing linear cooefficients of lines those segments lie on.
Parallel lines fullfill equation $ a_{AB} = a_{CD} $ and perpendicular lines fullfill equation $ a_{AB} * a_{CD} = -1 $. Adding that it is know how to calculate linear equation for line going through two point on a plane, we can derive equations:
$$ \frac{x_B-x_A}{y_B-y_A} = \frac{x_C - x_D}{y_C-y_D} $$
and
$$ \frac{x_B-x_A}{y_B-y_A}*\frac{x_C - x_D}{y_C-y_D} = -1 $$
However division is slow and innacurate then it is best to use them in different form:
$$ (x_B-x_A)(y_C-y_D) = (x_C - x_D)(y_B-y_A) $$
and
$$ (x_B-x_A)(x_C - x_D) = -(y_B-y_A)(y_C-y_D) $$.
And that is how those checks are implemented in functions
\subsection{Intersection point of segments}
TBD
\subsection{Area of a triangle}
If the triangle is right-angled then it's area can be calculated from equation $ P_\triangle = \frac{a*b}{2}$ where a and b are cathetuses. We only need to determine which of the sides these are - this can be done by checking perpendicularity of sides, what is already implemented (triangle's sides are segments).\\
Otherwise we can use Heron's equation that uses only lengths of the triangle's sides:
$$ P_\triangle = \sqrt{p(p-a)(p-b)(p-c)} $$
where:
$$ p = \frac{a+b+c}{2} $$
\subsection{Disjoint triangles}
TBD
\subsection{Hypothenuse in right-angled triangle}
To determine which of the triangle's sides is hypothenuse one only need to know which two sides are perpendicular, what is already implemented (triangle's sides are segments). After that length of hypothenuse can be calculated from Pythagoras Theorem: $ c = \sqrt{a^2+b^2} $.
\end{document}
